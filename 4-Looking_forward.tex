\section{Looking forward}
\label{sec-forward}
We have seen that adversarial machine learning can be naturally considered in the game theory framework. But, through our review, we left out some obvious questions, simply because I had not found relevant literature that tackle them directly. In this section, I will draw a picture of what I believe are interesting avenues of research at the intersection of adversarial machine learning and game theory. I will also attempt to argue how some of the results from the adversarial machine learning and game theory can be considered for the more general problem of out-of-distribution generalization.

\subsection{On modern machine learning and game theory}
In opposition with the field of physics where we have many theoretical results but no empirical demonstration to back them up, machine learning is in the interesting situation where our theory does not explain our empirical results. Most of the theoretical guarantees in learning theory require convex functions, or at the minimum linear function. However, recent machine learning models have hundred of composed piece-wise non-linearities. In this setup, not only we do not have a convex function, but we also have a non-linear function.

As we have seen during our review, Game Theory also rely on convex assumptions and linear functions. For this reason alone, all the results that we have seen do not readily transfer to modern machine learning. 